\documentclass[a4paper]{report}
\usepackage{graphicx}
\usepackage{xcolor}
\usepackage{colortbl}
\usepackage{float}
\usepackage{soul}
\usepackage{Sweave}
\usepackage{lscape}
\usepackage{microtype}
\usepackage{hyperref}
\usepackage[section]{placeins}
\usepackage{pslatex}
\usepackage{palatino}
\usepackage{avant}
\usepackage{verbatim}
%% layout commands may be written here
%% \newcounter{}
\begin{document}

\pagenumbering{arabic}



\title{Reproducible Reports with R , TEX , \& Sweave}
\author{Ivan C. Hanigan}
\date{\today}
\maketitle
\tableofcontents

\section*{Introduction}
This is a brief intro and template to Reproducible Reports with R , TEX , \& Sweave. For a great overview of why you might want to do this see: 
Scott, T. A. (n.d.). Reproducible Research with R , TEX , \& Sweave: A common ( flawed ) approach for generating statistical reports. Retrieved from \url{http://biostat.mc.vanderbilt.edu/wiki/Main/SweaveLatex}


For the header with palatino see 
\url{http://cran.r-project.org/web/packages/lazyWeave/lazyWeave.pdf}

For eg:
\begin{verbatim}
install.packages("lazyWeave")
require(lazyWeave)
lazy.file.start(docClass="report", 
packages=c("pslatex", "palatino", "avant"), 
title="Report Name", author="Your Name")
\end{verbatim}

For the intro to including R calculations in the paragraphs (ie using 'Sexpr') see:
\url{http://tex.stackexchange.com/a/22392}.

\section{Inline R output}
Say we want to calculate a number and report it we can use this code:

\begin{verbatim}
One and One is 2.
\end{verbatim}
Which will look like this: One and One is 2.

We might want to write multiple lines of R code:
\begin{verbatim}
Lm returns -0.322671000012202.
\end{verbatim}

Which will look like this: Lm returns -0.104693497041258.

\section{Inline R output with functions}
Including Functions can be tricky:
\begin{Schunk}
\begin{Sinput}
> sayhi <- function(k=2) {for(i in 1:k) cat("hi",i,"| ") }
> sayhi()
\end{Sinput}
\begin{Soutput}
hi 1 | hi 2 | 
\end{Soutput}
\end{Schunk}

But with Sexpr, {\tt sayhi()} prints <>.

Note that {\tt sayhi()} returns nothing; that's why Sexpr doesn't work.

\begin{Schunk}
\begin{Sinput}
> a <- sayhi()
\end{Sinput}
\begin{Soutput}
hi 1 | hi 2 | 
\end{Soutput}
\begin{Sinput}
> cat(a)
\end{Sinput}
\end{Schunk}

\begin{Schunk}
\begin{Sinput}
> savehi <- function(k=2) {
+   out <- c()
+   for(i in 1:k) {
+     out <- paste(out, "hi", i,"| ")
+   }
+   out
+ }
> hi <- savehi()
> cat(hi)
\end{Sinput}
\begin{Soutput}
 hi 1 |  hi 2 | 
\end{Soutput}
\end{Schunk}

And with Sexpr, {\tt savehi()} prints < hi 1 |  hi 2 | >.


% \documentclass{article}
% \begin{document}
\Sconcordance{concordance:sexpr.tex:sexpr.Rnw:%
1 40 1 1 2 1 0 3 1 22 0 2 2 1 0 1 1 17 0 1 2 5 1 1 2 1 0 1 1 4 0 1 2 25 %
1 1 2 1 0 1 1 6 0 1 2 8 1 1 2 6 0 1 1 3 0 1 2 2 1 1 8 7 0 2 1 6 0 1 2 %
20 1 1 2 25 0 1 2 1 1}

% \SweaveOpts{concordance=TRUE}
%% lazy.file.start(docClass = "report", packages = c("pslatex",      "palatino", "avant"), title = "Report Name", author = "Your Name")  

\end{document}
